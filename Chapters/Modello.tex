
%Metodo
%IL Metodo Sperimentale. FENOMENO MODELLO   ESPERIMENTO Osservazione degli oggetti e degli %strumenti a disposizione in laboratorio.
%Valutazione dell’incertezza sperimentale che ciascuna grandezza apporta alla misura di g, identificandole nelle richieste di precisione, accuratezza, sensibilità, per ciascuna misura / set di %misure.
%Confronto tra possibili metodi di analisi.
%Controlli sulla realizzazione pratica, raccolta e analisi dati (nessun prodotto di ricerca è migliore %del suo peggior fattore).

\section{Modello Teorico}
Per la stima del vettore accelerazione gravitazionale $\vec{g}$, utilizzeremo il moto armonico di un pendolo semplice.\\

In un pendolo semplice abbiamo un punto fisso, attorno al quale oscilla una piccola massa tenuta da un filo.\\
\begin{figure}[h]
    \centering
    \includegraphics[width=0.5\columnwidth]{Images/Modello teorico.png}
    \caption{Schema delle forze}
    \label{fig:modello teorico}
\end{figure}
Consideriamo il sistema di riferimento polare con origine nel punto fisso (in figura segnato con una \verb|x| di colore grigio), e angolo polare $\theta$, rispetto alla verticale.\\
Nell'immagine \figref{fig:modello teorico} sono rappresentate le forze che agiscono sul sistema.
Sulla massa $m$ agisce la forza peso $m\vec{g}$, e la tensione del filo $\Vec{T}$.
Il filo, lo considereremo inestensibile e di massa trascurabile.\\

Utilizzando la seconda legge di Newton troviamo l'equazione del moto:
\begin{equation}\label{equazione del moto}
    \pdv[2]{\theta}{t} + \frac{g}{\ell} \sin\theta = 0
\end{equation}
In cui $\ell$ è la lunghezza del filo e $t$ il tempo.\\

Tuttavia questa equazione non è analiticamente risolvibile, però possiamo utilizzare l'approssimazione: $\sin\theta \approx \theta$ per $0 \leq |\theta| \ll 0$, perciò:
\begin{equation}
    \pdv[2]{\theta}{t} = -\frac{g}{\ell}\theta
\end{equation}
che è l'equazione differenziale del moto armonico, da cui ricaviamo che:
$$\omega^2 = \frac{g}{\ell} \Rightarrow T = 2\pi \sqrt{\frac{\ell}{g}}$$
Dove $\omega$ è la pulsazione dell'oscillazione e $T$ il suo periodo.\\

\subsection{Correzioni}\label{correzioni}
\subsubsection*{Approssimazione \texorpdfstring{$\sin\theta\approx\theta$}{sin(x) = x}}
Per $\theta \approx 0$[rad] possiamo espandere il $\sin \theta$ con la serie di McLaurin: $\theta - \frac{1}{3!}\theta^3 + \smallo(\theta^4)$, sostituendo nella \eqref{equazione del moto} otteniamo:
$$    \pdv[2]{\theta}{t} = -\frac{g}{\ell}\theta + \frac{g}{\ell}\frac{1}{6}\theta^3$$
Da cui si può dimostrare che il periodo di oscillazione è pari a:
$$T = 4 \sqrt{\frac{\ell}{g}}K$$
In cui $K$ è l'integrale ellittico completo di prima specie valutato in $\sin^2\qty(\frac{\theta_\text{max}}{2})$, il periodo è quindi:
$$ T = 2\pi \sqrt{\frac{l}{g}}\qty(1+\frac{\theta_\text{max}^2}{16} + \smallo(\theta_\text{max}^3))$$

\subsubsection*{Attrito}
Con l'attrito nell'equazione del moto \eqref{equazione del moto} si aggiunge il temine di smorzamento viscoso, dato dalla relazione: $F_\text{smorzamento} = -2\gamma v$, in cui $\gamma$ è il coefficiente di smorzamento e $v$ è la velocità.\\ 
Perciò otteniamo l'equazione del moto:
$$\pdv[2]{\theta}{t} +\gamma\frac{\ell}{m}\pdv{\theta}{t} + \frac{g}{\ell} \theta = 0$$
Da cui otteniamo che il periodo è:
$$T^2 = 4\pi^2 \frac{\ell}{g} \qty(1 + \frac{\gamma^2}{4\pi^2}) = T_0^2\qty(1 + \frac{\gamma^2}{4\pi^2})$$
In cui $T_0$ è il periodo senza attrito.

\subsubsection*{Momento di inerzia}
Siccome il pendolo semplice è un modello ideale non riproducibile nella realtà, dobbiamo considerare anche la correzione del momento di inerzia.\\
Considerando che la massa del nostro pendolo è di forma cilindrica possiamo applicare il teorema di Huygens-Steiner e ricaviamo la correzione del momento di inerzia, che dobbiamo applicare sia al periodo, sia sulla lunghezza del pendolo.\\
$$T = \sqrt{\frac{I}{mgL_0}} = \sqrt{\frac{\ell_\text{eq}}{g}}$$
$$\ell_\text{eq} = \ell_0 + \frac{4R^2 + h^2}{16\ell_0}$$
In cui $I$ è il momento d'inerzia, $\ell_\text{eq}$ la lunghezza equivalente, $\ell_0$ la lunghezza reale, $R$ il raggio del cilindro e $h$ l'altezza del cilindro.

\subsubsection*{Valore di \texorpdfstring{$g$}{g}}
Perciò considerando tutte le correzioni il valore dell'accelerazione gravitazionale diventa:
\begin{align*}
g_\text{reale} = g_\text{moto armonico} 
&+\Delta g_\text{angolo max} \\
&+ \Delta g_\text{attrito} \\
&+ \Delta g_\text{inerzia}
\end{align*}
In cui $\Delta g$ sono le varie correzioni applicate.\\