\section{Abstract}
In questo esperimento misureremo il valore del modulo dell'accelerazione gravitazionale locale $g$, tramite un pendolo. Misureremo direttamente la lunghezza e il periodo del pendolo, per poi effettuare una regressione lineare per ricavare il valore di $g$. Con diversi strumenti e diversi metodi abbiamo ottenuto che il valore di $g$ fosse nei casi con incertezza più grossolana compatibile con il valore atteso si $9.806$m/s$^2$, e nei casi con incertezza più fine non fosse compatibile.\\
In particolare utilizzando la regressione lineare ottenuta con il fototraguardo abbiamo ottenuto $g=9.786\pm0.002$m/s$^2$, eseguendo un t-test di compatibilità con il valore atteso risulta avere $p$-value: 0.00002\%, ovvero di non essere compatibile con il valore atteso. Calcolando il $\chi^2$ troviamo che esso è pari a 437, da cui deduciamo che i nostri set di misura non seguono un andamento lineare all'incertezza del fototraguardo, ciò è probabilmente dovuto agli allungamenti del filo, indesiderati, che abbiamo osservato.\\