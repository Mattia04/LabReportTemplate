\section{Introduzione}
L'obbiettivo di questo esperimento è di determinare il valore del modulo del vettore accelerazione gravitazionale $g \coloneq |\Vec{g}|$, utilizzando un pendolo.\\
Misureremo direttamente la lunghezza del pendolo e il periodo di oscillazione, da una regressione lineare è possibile estrarre il valore del modulo dell'accelerazione gravitazionale locale.\\

Considereremo in particolare i termini correttivi di: inerzia, attrito e piccole oscillazioni.\\

Misureremo i periodi con due strumenti diversi e confronteremo i risultati ottenuti.\\

Infine confronteremo i risultati, da noi ottenuti, con il valore che ci aspettiamo dell'accelerazione gravitazionale: 9.806m/s$^2$.
