\section{Strumentazione}\label{strumentazione}
\subsection*{Montaggio}\label{montaggio}
\begin{figure}[h]
    \centering
    \includegraphics[width=0.9\columnwidth]{Images/Strumentazione_Prospettiva_accgrav.png}
    \caption{Strumentazione utilizzata, in prospettiva.}
    \label{fig:strumentazione}
\end{figure}
\begin{figure}[h]
    \centering
    \includegraphics[height=1.5in]{Images/Strumentazione_Fronte_accgrav.png}
    \vline
    \vspace{2pt}
    \includegraphics[height=1.5in]{Images/Strumentazione_Lato_accgrav.png}
    \caption{Strumentazione utilizzata, dal fronte (sinistra) e dal lato (destra).}
    \label{fig:strumentazione 2}
\end{figure}

\subsubsection*{Struttura di supporto}
In \figref{fig:strumentazione} e \figref{fig:strumentazione 2}\footnote{Nella \figref{fig:strumentazione 2}, dal lato, non abbiamo disegnato il goniometro per complicare ulteriormente la figura.} è illustrato uno schema della strumentazione utilizzata.\\

Per costruire la struttura di supporto abbiamo fissato un'asta metallica \{1\} al banco di lavoro \{2\} in verticale, tramite un morsetto \{3\}. \\
All'asta verticale è stata fissata un'asta orizzontale \{4\} tramite un secondo morsetto. \\
Per aumentare la stabilità del sistema abbiamo fissato una terza asta \{5\}, inclinata di circa 45° rispetto all'orizzontale, al banco di lavoro \{6\} e all'asta orizzontale.\\
In questo modo la struttura di supporto è sufficientemente rigida da non oscillare lungo il piano $\Vec{y}\times\Vec{z}$ durante il moto del pendolo\footnote{La struttura potrebbe comunque oscillare lungo l'asse $\Vec{x}$, perciò abbiamo tenuto il pendolo il più vicino possibile alle due aste, per ridurre al minimo tale oscillazione.}.

\subsubsection*{Pendolo}
Per quanto riguarda il pendolo, abbiamo fissato all'asta orizzontale \{4\} i due estremi di un filo \{7\}, tenendoli a una distanza ravvicinata (circa 2cm) dall'asta orizzontale e obliqua, e mantenendo circa 5cm tra i due estremi del filo.\\
Il filo deve essere sufficientemente spesso da non avere proprietà elastiche\footnote{Il filo è inestensibile} ma sufficientemente sottile da non avere proprietà rigide\footnote{Il filo non si oppone al moto del pendolo.}.\\

Al centro del filo abbiamo posizionato il gancio con le masse \{8\}, in questo punto poi andremo ad aggiungere pezzi di filo \{9\} per allungare la lunghezza del pendolo (questa procedura verrà spiegata meglio nella sezione \nameref{misure effettuate}).\\

La lunghezza totale del pendolo così costruito, per questioni pratiche, deve essere tale che il punto più basso della massa appesa sia entro la portata dello strumento di misura, il calibro (portata massima: 230.00mm, risoluzione: 0.02mm), da un riferimento fisso, nel nostro caso il banco di lavoro \{2\}.\\
Il gancio con le masse utilizzate deve essere di un materiale con densità elevata (noi abbiamo utilizzato masse con densità di circa 7.8g/cm$^3$), per un totale di circa 100g.\\

\subsubsection*{Apparecchiatura di misura}
Per il fototraguardo (risoluzione: 0.0001s), abbiamo fissato la sua asta di supporto \{10\} al banco di lavoro con un morsetto, in modo tale che l'asta di supporto sia posizionata (in verticale) sotto l'asta orizzontale \{4\}. Il fototraguardo \{11\} l'abbiamo montato sulla sua asta di supporto, così che possa essere spostato verticalmente tra i vari set di misura.\\
Inoltre abbiamo posizionato un goniometro \{12\} sull'asta orizzontale per conoscere l'angolo massimo di oscillazione.\\
