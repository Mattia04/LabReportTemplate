\section{Analisi Dati}\label{analisi dati}
Nel file \verb|script.ipynb| sulla \href{https://github.com/Mattia04/Accelerazione-gravitazionale/tree/main}{Repository:\\Accelerazione-gravitazionale} è contenuto il codice python utilizzato per fare l'analisi statistica, inoltre tutte le immagini\footnote{Le immagini dell'analisi dati sono in formato vettoriale, quindi se ci fossero difficoltà ad osservarle si possono ingrandire a piacere.} sono salvate nella directory \verb|images\|.\\

\subsection*{Correzione del momento d'inerzia}
La correzione del momento di inerzia incide sul periodo del pendolo, tuttavia possiamo utilizzare la lunghezza equivalente, andando così a correggere il valore della lunghezza del pendolo, come mostrato nella sezione \ref{correzioni}.\\
Per calcolare la lunghezza reale però abbiamo prima dovuto trasformare le lunghezze misurate con il calibro nella lunghezza reale del pendolo $\ell_{0,i}$ tramite la formula:
$$\ell_{0,i} = \ell_\text{m} - CM + \Delta\ell_{\text{rif},i}$$
In cui $\Delta\ell_{\text{rif},i}$ è la differenza dalla misura  iniziale ($\ell_\text{rif} = 206,10$mm) alla $i$-esima misura effetuata con il calibro, $\ell_\text{m}$ è la lunghezza del pendolo misurata con il metro, e il $CM$ è il centro di massa calcolato dalla formula: \smash{$CM = \frac{\sum_j m_j L_j}{m_j}$}, in cui $m_j$ e $L_j$ sono le masse e posizioni dei singoli componenti del sistema di masse, che risulta essere pari a 23.78mm per tutte le misure effettuate.\\
A $\ell_{0}$ abbiamo quindi applicato la correzione del momento di inerzia: $$\ell_\text{eq} = \ell_0 + \frac{4R^2 + h^2}{16\ell_0}$$

\subsection*{Correzione per angoli piccoli}\label{correzione angoli piccoli}
Riprendendo la formula espressa in \ref{correzioni}:
$$ T = 2\pi \sqrt{\frac{l}{g}}\qty(1+\frac{\theta_\text{max}^2}{16}) = T_0 \qty(1+\frac{\theta_\text{max}^2}{16})$$
Considerando un angolo di partenza di 3°, quindi $\theta_\text{max} = 3[deg] \approx 0,052[rad]$, otteniamo che la differenza tra il periodo osservato $T$ e il periodo armonico $T_0$ è pari a $T- T_0 = 0,00017T_0$.\\

\subsection*{Regressione lineare}
Per il calcolo di $g$ abbiamo effettuato una regressione lineare tra i periodi al quadrato e le relative lunghezze del pendolo, utilizzando la formula: 
$$T^2 = \frac{4\pi^2}{g}\ell = A\ell + B$$
Dalla quale ricaviamo $g = \frac{4 \pi^2}{A}$, $\sigma_g = \frac{4\pi^2 \sigma_A}{A^2}$, e troviamo che $\sigma_{T^2} = 2T\sigma_T$ e $\sigma_{\ell, \text{prop.}} = A\sigma_\ell$, infine ci aspettiamo che il parametro $B$ sia circa zero.\\
Per ottenere un $A$-test con cui propagare le incertezze della lunghezza sui periodi abbiamo considerato la prima e l'ultima lunghezza e abbiamo calcolato che il parametro $A \approx 0.004\text{s}^2/\text{mm}$ per tutti i set.\\
Quindi abbiamo propagato l'errore sulle lunghezze sui periodi, tramite somma in quadratura di $\sigma_{T^2}$ e $\sigma_{\ell, \text{prop.}}$, e così abbiamo potuto svolgere una regressione lineare pesata.\\


\subsection*{Media mobile}
Per minimizzare l’incertezza sul periodo, abbiamo applicato la media mobile su ogni set di misure effettuato con finestra di M=39 (circa 2/3N, in cui N=59, è il numero delle misure per ogni set), abbiamo potuto utilizzare la media mobile perché le misure effettuate sono \textit{set storici}, ovvero le misure sono state effettuate in modo contiguo e quindi ogni singolo valore del periodo è influenzato dall'errore del periodo precedente e di quello successivo.\\
Sia $K = N - M +1$ il numero di finestre considerate.\\
Allora formule utilizzate sono le seguenti:
$$\langle T_\text{MM} \rangle = \frac{1}{K}\sum_{k=1}^{K}\langle T_k\rangle = 
\frac{1}{K}\frac{1}{M}\sum_{k=1}^{K}\sum_{i=1}^{M} T_{k+i} $$
$$\sigma_\text{MA} = \frac{\langle \sigma_k \rangle}{M\sqrt{K}}$$
In cui $\langle T_\text{MM} \rangle$ è il periodo medio del set di misure, $\langle T_k\rangle$ è il periodo medio di una finestra di misure, e $T_j$ è il periodo della misura $j$-esima, l'incertezza della media mobile è $\sigma_\text{MA}$.\\


\subsection{Cronometro}
Per ogni set di misure abbiamo calcolato il periodo medio con media semplice e media mobile, che risulta essere uguale per entrambi i casi, su cui è stata applicata la correzione per angoli piccoli.\\

In \tabref{tab:misure cronometro mattia} sono riportate i valori dei periodi medi, prese con il cronometro dallo sperimentatore 1, alle varie lunghezze del pendolo. La relativa incertezza è stata calcolata con media semplice (\textbf{MS}) e media mobile (\textbf{MM}).\\
\begin{table}[h]
    \centering
    \begin{tabular}{r|c|c|c}
        \textbf{L} [mm] & \textbf{Periodi} [s]& $\mathbf{\sigma_{T}}$\textbf{ MS} [s] & $\mathbf{\sigma_{T}}$\textbf{ MM} [s] \\
        \hline
        381.40 & 1.2395 & $\pm$ 0.009& $\pm$ 0.0003 \\
        414.91 & 1.2930 & $\pm$ 0.007& $\pm$ 0.0003 \\
        460.01 & 1.3614 & $\pm$ 0.006& $\pm$ 0.0002 \\
        488.40  & 1.4022 & $\pm$ 0.006& $\pm$ 0.0002 \\
        519.89  & 1.4474 & $\pm$ 0.007& $\pm$ 0.0003 \\
        558.24  & 1.4977 & $\pm$ 0.006& $\pm$ 0.0002 \\
    \end{tabular}
    \caption{Misure effettuate con il cronometro, dello sperimentatore 1, le lunghezze (\textbf{L}) hanno tutte incertezza $\pm$0.02mm}
    \label{tab:misure cronometro mattia}
\end{table}

I valori dei periodi per i set degli sperimentatori 2 e 3 non sono stati inseriti, sono però consultabili, nella sezione ``Analisi, per il cronometro" sul file \verb|script.ipynb|.\\

Abbiamo poi effettuato una regressione lineare (\figref{fig:reg lin cronometro}) sulle misure effettuate di ogni sperimentatore, e abbiamo ottenuto i valori dei parametri $A$ e $B$, riportati in \tabref{tab:valori regressione}, da cui abbiamo potuto ricavare i valori dell'accelerazione gravitazione $g$ mostrati in \tabref{tab:valori g cronometro}.\\

\begin{table}[h]
\centering
\begin{tabular}{c|c|c}
    \textbf{exp} & \textbf{Metodo} & \textbf{Parametri} \\
    \hline
    \multirow{4}{*}{\textbf{1}} & \multirow{2}{*}{MS} & \( A = 0.00402 \pm 0.00014 \) \\
                            &    & \( B = 0.005 \pm 0.064 \) \\
    \cline{2-3}
                            & \multirow{2}{*}{MM}& \( A = 0.004018 \pm 0.000005 \) \\
                            &    & \( B = 0.0054 \pm 0.0024 \) \\
    \hline
    \multirow{4}{*}{\textbf{2}} & \multirow{2}{*}{MS} & \( A = 0.00401 \pm 0.00012 \) \\
                             &    & \( B = 0.010 \pm 0.056 \) \\
    \cline{2-3}
                             & \multirow{2}{*}{MM} & \( A = 0.003982 \pm 0.000004 \) \\
                             &    & \( B = 0.0225 \pm 0.0019 \) \\
    \hline
    \multirow{4}{*}{\textbf{3}} & \multirow{2}{*}{MS} & \( A = 0.00398 \pm 0.00011 \) \\
                             &    & \( B = 0.021 \pm 0.053 \) \\
    \cline{2-3}
                             & \multirow{2}{*}{MM} & \( A = 0.003993 \pm 0.000004 \) \\
                             &    & \( B = 0.0186 \pm 0.0018 \) \\
\end{tabular}
\caption{Parametri della regressione per ciascun set del cronometro e metodo.\\
$A$ ha unità di misura $\text{s}^2/\text{mm}$ e $B$ ha unità di misura $\text{s}^2$.}
\label{tab:valori regressione}
\end{table}

\begin{table}[h]
    \centering
    \begin{tabular}{r|c|c}
        \textbf{exp} & $g$ \textbf{(MS)} [$\text{m/s}^2$]& $g$ \textbf{(MM) [$\text{m/s}^2$]}\\
        \hline
        \textbf{1} & $9.82\pm0.33 $ & $9.824\pm0.013$ \\
        \textbf{2} & $9.85\pm0.30 $ & $9.913\pm0.010$ \\
        \textbf{3} & $9.91\pm0.28 $ & $9.888\pm0.010$ \\
    \end{tabular}
    \caption{Tabella dei valori di $g$ ottenuti con il cronometro dagli sperimentatori (exp)}
    \label{tab:valori g cronometro}
\end{table}

Possiamo ora calcolare la p-value riferita alla confidenza con cui i valori di $g$ ottenuti dai vari metodi e sperimentatori siano compatibili con il valore atteso 9.806m/s$^2$, i p-value sono indicati in \tabref{tab:p_values}, inoltre abbiamo anche controllato se il parametro $B$ della regressione lineare fosse compatibile con lo zero\footnote{Nel calcolare la compatibilità di $B$ sullo zero abbiamo considerato anche il fatto che la lunghezza totale è stata misurata con il metro, con risoluzione di 1mm, tale incertezza è stata sommata in quadratura all'incertezza di $B$ della regressione $\sigma_B$.}.\\
Dai $p$-value notiamo che, i valori di $g$ calcolati con il cronometro e media semplice sono compatibili (p-value $> 5$\%) con il valore atteso e con lo zero, tuttavia utilizzando la media mobile, i valori per gli sperimentatori 2 e 3 sono incompatibili. In particolare l'incompatibilità con lo zero ci suggerisce la presenza di un errore sistematico, ipotizziamo sia causato dall'allungamento del filo durante il corso dell'esperimento.\\ 

\begin{table}[h]
\centering
\begin{tabular}{c|c|c|c}
    \textbf{exp} & \textbf{Metodo} & $p(g = 9.806$m/s$^2)$ & $p(B = 0)$ \\
    \hline
    \multirow{2}{*}{\textbf{1}} & MS & 97\% & 94\% \\
                            & MM & 21\% & 9.6\% \\
    \hline
    \multirow{2}{*}{\textbf{2}} & MS & 88\% & 86\% \\
                             & MM & 0.04\% & 0.02\% \\
    \hline
    \multirow{2}{*}{\textbf{3}} & MS & 73\% & 70\% \\
                             & MM & 0.10\% & 0.20\% \\
\end{tabular}
\caption{p-value di compatibilità con \( g \) attesa e p-value di compatibilità di $B$ con lo zero, per ciascun sperimentatore e metodo.}
\label{tab:p_values}
\end{table}

\begin{figure}[h]
    \centering
    \includesvg[width=\columnwidth]{Images/analisi/Regressione Lineare Cronometro.svg}
    \caption{Regressione lineare (RL) per le misure effettuate con il cronometro; sopra utilizzando la media semplice, sotto utilizzando la media mobile.}
    \label{fig:reg lin cronometro}
\end{figure}


\subsubsection*{Valore di \texorpdfstring{$\chi^2$}{chi quadro}}
Calcolando i $\chi^2$ della regressione lineare per ogni sperimentatore otteniamo i valori indicati in \tabref{tab:chi2red_values}, i valori più alti indicano un andamento non lineare. Come ci aspettavamo, i valori di $\chi^2_\text{rid}$ della media mobile sono molto più alti rispetto a quelli ottenuti con la media semplice.

\begin{table}[h]
\centering
\begin{tabular}{c|c|c}
exp & Metodo & \(\chi^2_\text{rid}\) \\
\hline
\multirow{2}{*}{1} & MS & 0.14 \\
                        & MM & 10.3 \\
\hline
\multirow{2}{*}{2} & MS & 0.05 \\
                         & MM & 2.35 \\
\hline
\multirow{2}{*}{3} & MS & 0.011 \\
                         & MM & 8.01 \\
\end{tabular}
\caption{Valori di \(\chi^2_{red}\) per ciascun sperimentatore e metodo.}
\label{tab:chi2red_values}
\end{table}

\subsection{Fototraguardo}
Per quanto riguarda le misure effettuate con il fototraguardo, siccome il numero di set di misura per ogni sperimentatore è insufficiente a fare 3 regressioni lineari abbiamo deciso di fare una regressione lineare sola, con tutti i set di misure a disposizione.\\

Come con il cronometro, abbiamo calcolato per ogni set il periodo medio, stavolta utilizzando unicamente la media semplice, siccome il fototraguardo ha risoluzione molto inferiore, utilizzare la media mobile non ha nessun fine pratico.\\

I valori dei periodi, con relativa incertezza, per tutti i set non sono stati inseriti, sono però consultabili, nella sezione "Analisi, per il fototraguardo" sul file \verb|script.ipynb|.\\

Con i valori dei periodi e le relative lunghezze del pendolo abbiamo effettuato una regressione lineare, mostrata in \figref{fig:reg lin fototraguardo}, ottenendo i valori dei parametri: $A=0.004034 \pm 0.000004\text{s}^2/\text{mm}$, $B=0.0007 \pm0.0018\text{s}^2$.\\
Dai parametri abbiamo ricavato il valore di $g=9.786 \pm0.002$m/s$^2$.\\

Allo stesso modo di come abbiamo fatto per il cronometro, abbiamo calcolato le p-value, tramite t-Student, di compatibilità con il valore atteso e con lo zero della regressione, ottenendo: $p$(\(g = 9.806\)m/s$^2$) = 0.00002\%, e $p$($B = 0$) = 69\%; quindi il parametro $B$ è compatibile con zero, ma non è compatibile il valore di $g$ da noi ottenuto, con il valore atteso.\\
E il valore del $\chi^2_\text{rid} = 437$ della regressione lineare, ci indica che all'incertezza del fototraguardo la regressione ha un andamento sub-lineare.\\

\begin{figure}[h]
    \centering
    \includesvg[width=\columnwidth]{Images/analisi/Regressione Lineare Fototraguardo.svg}
    \caption{Regressione lineare per i set di misure effettuate con il fototraguardo.}
    \label{fig:reg lin fototraguardo}
\end{figure}

\subsubsection*{Attrito}
Utilizzando la misura con il fototraguardo effettuata con angoli differenti (mostrata in \figref{fig:fit angoli diversi}), tramite un fit lineare abbiamo stimato che il coefficiente di attrito $\gamma$, per la misura effettuata partendo da 15° è pari a $0.00502 \pm0.00008$, utilizzando la formula mostrata in \ref{correzioni}: $$T^2 = T_0^2\qty(1 + \frac{\gamma^2}{4\pi^2})$$
Risulta che $T = T_0$ fino al settimo decimale, perciò questa differenza è molto inferiore all'incertezza sui singoli periodi (l'incertezza è al terzo decimale), e quindi è trascurabile l'effetto dell'attrito per la misura con 15°. Siccome maggiore è l'angolo di partenza, maggiore sarà l'attrito viscoso, allora sarà trascurabile in tutte le altre misure effettuate con angoli di partenza inferiori.\\

\begin{figure}[h]
    \centering
    \includesvg[width=\columnwidth]{Images/analisi/Fit Esponenziale per Angoli Diversi.svg}
    \caption{Fit esponenziale ad angoli iniziali diversi, per la stima dell'attrito viscoso.}
    \label{fig:fit angoli diversi}
\end{figure}

\subsubsection*{Osservazione sulle misure effettuate}
Abbiamo notato che le misure effettuate per 3° e 7° presentano battimenti, probabilmente sono dovuti al fatto che le oscillazioni del pendolo erano ``spurie" e quindi hanno impiegato più oscillazioni a stabilizzarsi.\\

%Siccome il set di misure utilizzato per calcolare l'attrito con il fototraguardo abbiamo pensato, di inserire il valore all'asintoto degli esponenziali per i differenti angoli, che dovrebbe corrispondere al valore 

%\begin{figure}[h]
%    \centering
%    \includesvg[width=\columnwidth]{Images/analisi/Regressione Lineare Fototraguardo2.svg}
%    \caption{Regressione lineare per le misure effettuate con il fototraguardo, aggiungendo anche le misure effettuate con angoli diversi.}
%    \label{fig:reg lin fototraguardo unito}
%\end{figure}
